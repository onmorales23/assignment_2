% Options for packages loaded elsewhere
\PassOptionsToPackage{unicode}{hyperref}
\PassOptionsToPackage{hyphens}{url}
%
\documentclass[
]{article}
\usepackage{amsmath,amssymb}
\usepackage{lmodern}
\usepackage{iftex}
\ifPDFTeX
  \usepackage[T1]{fontenc}
  \usepackage[utf8]{inputenc}
  \usepackage{textcomp} % provide euro and other symbols
\else % if luatex or xetex
  \usepackage{unicode-math}
  \defaultfontfeatures{Scale=MatchLowercase}
  \defaultfontfeatures[\rmfamily]{Ligatures=TeX,Scale=1}
\fi
% Use upquote if available, for straight quotes in verbatim environments
\IfFileExists{upquote.sty}{\usepackage{upquote}}{}
\IfFileExists{microtype.sty}{% use microtype if available
  \usepackage[]{microtype}
  \UseMicrotypeSet[protrusion]{basicmath} % disable protrusion for tt fonts
}{}
\makeatletter
\@ifundefined{KOMAClassName}{% if non-KOMA class
  \IfFileExists{parskip.sty}{%
    \usepackage{parskip}
  }{% else
    \setlength{\parindent}{0pt}
    \setlength{\parskip}{6pt plus 2pt minus 1pt}}
}{% if KOMA class
  \KOMAoptions{parskip=half}}
\makeatother
\usepackage{xcolor}
\usepackage[margin=1in]{geometry}
\usepackage{color}
\usepackage{fancyvrb}
\newcommand{\VerbBar}{|}
\newcommand{\VERB}{\Verb[commandchars=\\\{\}]}
\DefineVerbatimEnvironment{Highlighting}{Verbatim}{commandchars=\\\{\}}
% Add ',fontsize=\small' for more characters per line
\usepackage{framed}
\definecolor{shadecolor}{RGB}{248,248,248}
\newenvironment{Shaded}{\begin{snugshade}}{\end{snugshade}}
\newcommand{\AlertTok}[1]{\textcolor[rgb]{0.94,0.16,0.16}{#1}}
\newcommand{\AnnotationTok}[1]{\textcolor[rgb]{0.56,0.35,0.01}{\textbf{\textit{#1}}}}
\newcommand{\AttributeTok}[1]{\textcolor[rgb]{0.77,0.63,0.00}{#1}}
\newcommand{\BaseNTok}[1]{\textcolor[rgb]{0.00,0.00,0.81}{#1}}
\newcommand{\BuiltInTok}[1]{#1}
\newcommand{\CharTok}[1]{\textcolor[rgb]{0.31,0.60,0.02}{#1}}
\newcommand{\CommentTok}[1]{\textcolor[rgb]{0.56,0.35,0.01}{\textit{#1}}}
\newcommand{\CommentVarTok}[1]{\textcolor[rgb]{0.56,0.35,0.01}{\textbf{\textit{#1}}}}
\newcommand{\ConstantTok}[1]{\textcolor[rgb]{0.00,0.00,0.00}{#1}}
\newcommand{\ControlFlowTok}[1]{\textcolor[rgb]{0.13,0.29,0.53}{\textbf{#1}}}
\newcommand{\DataTypeTok}[1]{\textcolor[rgb]{0.13,0.29,0.53}{#1}}
\newcommand{\DecValTok}[1]{\textcolor[rgb]{0.00,0.00,0.81}{#1}}
\newcommand{\DocumentationTok}[1]{\textcolor[rgb]{0.56,0.35,0.01}{\textbf{\textit{#1}}}}
\newcommand{\ErrorTok}[1]{\textcolor[rgb]{0.64,0.00,0.00}{\textbf{#1}}}
\newcommand{\ExtensionTok}[1]{#1}
\newcommand{\FloatTok}[1]{\textcolor[rgb]{0.00,0.00,0.81}{#1}}
\newcommand{\FunctionTok}[1]{\textcolor[rgb]{0.00,0.00,0.00}{#1}}
\newcommand{\ImportTok}[1]{#1}
\newcommand{\InformationTok}[1]{\textcolor[rgb]{0.56,0.35,0.01}{\textbf{\textit{#1}}}}
\newcommand{\KeywordTok}[1]{\textcolor[rgb]{0.13,0.29,0.53}{\textbf{#1}}}
\newcommand{\NormalTok}[1]{#1}
\newcommand{\OperatorTok}[1]{\textcolor[rgb]{0.81,0.36,0.00}{\textbf{#1}}}
\newcommand{\OtherTok}[1]{\textcolor[rgb]{0.56,0.35,0.01}{#1}}
\newcommand{\PreprocessorTok}[1]{\textcolor[rgb]{0.56,0.35,0.01}{\textit{#1}}}
\newcommand{\RegionMarkerTok}[1]{#1}
\newcommand{\SpecialCharTok}[1]{\textcolor[rgb]{0.00,0.00,0.00}{#1}}
\newcommand{\SpecialStringTok}[1]{\textcolor[rgb]{0.31,0.60,0.02}{#1}}
\newcommand{\StringTok}[1]{\textcolor[rgb]{0.31,0.60,0.02}{#1}}
\newcommand{\VariableTok}[1]{\textcolor[rgb]{0.00,0.00,0.00}{#1}}
\newcommand{\VerbatimStringTok}[1]{\textcolor[rgb]{0.31,0.60,0.02}{#1}}
\newcommand{\WarningTok}[1]{\textcolor[rgb]{0.56,0.35,0.01}{\textbf{\textit{#1}}}}
\usepackage{graphicx}
\makeatletter
\def\maxwidth{\ifdim\Gin@nat@width>\linewidth\linewidth\else\Gin@nat@width\fi}
\def\maxheight{\ifdim\Gin@nat@height>\textheight\textheight\else\Gin@nat@height\fi}
\makeatother
% Scale images if necessary, so that they will not overflow the page
% margins by default, and it is still possible to overwrite the defaults
% using explicit options in \includegraphics[width, height, ...]{}
\setkeys{Gin}{width=\maxwidth,height=\maxheight,keepaspectratio}
% Set default figure placement to htbp
\makeatletter
\def\fps@figure{htbp}
\makeatother
\setlength{\emergencystretch}{3em} % prevent overfull lines
\providecommand{\tightlist}{%
  \setlength{\itemsep}{0pt}\setlength{\parskip}{0pt}}
\setcounter{secnumdepth}{-\maxdimen} % remove section numbering
\usepackage{booktabs}
\usepackage{longtable}
\usepackage{array}
\usepackage{multirow}
\usepackage{wrapfig}
\usepackage{float}
\usepackage{colortbl}
\usepackage{pdflscape}
\usepackage{tabu}
\usepackage{threeparttable}
\usepackage{threeparttablex}
\usepackage[normalem]{ulem}
\usepackage{makecell}
\usepackage{xcolor}
\ifLuaTeX
  \usepackage{selnolig}  % disable illegal ligatures
\fi
\IfFileExists{bookmark.sty}{\usepackage{bookmark}}{\usepackage{hyperref}}
\IfFileExists{xurl.sty}{\usepackage{xurl}}{} % add URL line breaks if available
\urlstyle{same} % disable monospaced font for URLs
\hypersetup{
  pdftitle={Data Analysis 2},
  pdfauthor={Ziying Li, Matt Capaldi, Yujuan Gao, Olivia Morales},
  hidelinks,
  pdfcreator={LaTeX via pandoc}}

\title{Data Analysis 2}
\author{Ziying Li, Matt Capaldi, Yujuan Gao, Olivia Morales}
\date{\today}

\begin{document}
\maketitle

\usepackage{fontspec}

\begin{Shaded}
\begin{Highlighting}[]
\DocumentationTok{\#\# libraries}
\NormalTok{libs }\OtherTok{\textless{}{-}} \FunctionTok{c}\NormalTok{(}\StringTok{"tidyverse"}\NormalTok{, }\StringTok{"haven"}\NormalTok{, }\StringTok{"bibtex"}\NormalTok{, }\StringTok{"psych"}\NormalTok{, }\StringTok{"knitr"}\NormalTok{, }\StringTok{"pastecs"}\NormalTok{, }\StringTok{"kableExtra"}\NormalTok{,}\StringTok{"survey"}\NormalTok{, }\StringTok{"cobalt"}\NormalTok{, }\StringTok{"randomForest"}\NormalTok{, }\StringTok{"ipred"}\NormalTok{,}\StringTok{"rpart"}\NormalTok{, }\StringTok{"baguette"}\NormalTok{, }\StringTok{"parsnip"}\NormalTok{, }\StringTok{"SimDesign"}\NormalTok{, }\StringTok{"bartCause"}\NormalTok{, }\StringTok{"lme4"}\NormalTok{, }\StringTok{"grf"}\NormalTok{, }\StringTok{"GenericML"}\NormalTok{)}

\FunctionTok{sapply}\NormalTok{(libs, require, }\AttributeTok{character.only =} \ConstantTok{TRUE}\NormalTok{)}
\end{Highlighting}
\end{Shaded}

\begin{verbatim}
##    tidyverse        haven       bibtex        psych        knitr      pastecs 
##         TRUE         TRUE         TRUE         TRUE         TRUE         TRUE 
##   kableExtra       survey       cobalt randomForest        ipred        rpart 
##         TRUE         TRUE         TRUE         TRUE         TRUE         TRUE 
##     baguette      parsnip    SimDesign    bartCause         lme4          grf 
##         TRUE         TRUE         TRUE         TRUE         TRUE         TRUE 
##    GenericML 
##         TRUE
\end{verbatim}

\begin{Shaded}
\begin{Highlighting}[]
\DocumentationTok{\#\# directory path (assignment\_2 as current working directory)}

\NormalTok{data\_dir }\OtherTok{\textless{}{-}} \FunctionTok{file.path}\NormalTok{(}\StringTok{"."}\NormalTok{, }\StringTok{"data"}\NormalTok{)}

\DocumentationTok{\#\# loading data}
\FunctionTok{load}\NormalTok{(}\FunctionTok{file.path}\NormalTok{(data\_dir, }\StringTok{"chapter\_10\_data\_cleaned\_and\_imputed.Rdata"}\NormalTok{))}

\NormalTok{train }\OtherTok{\textless{}{-}}\NormalTok{ data }\SpecialCharTok{\%\textgreater{}\%} 
  \FunctionTok{sample\_frac}\NormalTok{(}\AttributeTok{size =} \FloatTok{0.5}\NormalTok{)}

\NormalTok{test }\OtherTok{\textless{}{-}} \FunctionTok{anti\_join}\NormalTok{(data, train)}

\NormalTok{train }\OtherTok{\textless{}{-}}\NormalTok{ train }\SpecialCharTok{\%\textgreater{}\%} \FunctionTok{mutate}\NormalTok{(}\FunctionTok{across}\NormalTok{(}\AttributeTok{.fns =}\NormalTok{ as.numeric))    }
\NormalTok{test }\OtherTok{\textless{}{-}}\NormalTok{ test }\SpecialCharTok{\%\textgreater{}\%} \FunctionTok{mutate}\NormalTok{(}\FunctionTok{across}\NormalTok{(}\AttributeTok{.fns =}\NormalTok{ as.numeric))    }
\end{Highlighting}
\end{Shaded}

\hypertarget{definingestimating-propensity-score-model}{%
\section{Defining/Estimating Propensity Score
Model}\label{definingestimating-propensity-score-model}}

\begin{Shaded}
\begin{Highlighting}[]
\NormalTok{covariateNames }\OtherTok{\textless{}{-}} \FunctionTok{c}\NormalTok{(}
    \StringTok{"X1RTHETK1"}\NormalTok{, }
    \StringTok{"X1MTHETK1"}\NormalTok{,}
    \StringTok{"X1TCHAPP"}\NormalTok{, }
    \StringTok{"X1TCHCON"}\NormalTok{,}
    \StringTok{"X1TCHPER"}\NormalTok{, }
    \StringTok{"X1TCHEXT"}\NormalTok{,}
    \StringTok{"X1TCHINT"}\NormalTok{,}
    \StringTok{"X1ATTNFS"}\NormalTok{,}
    \StringTok{"X1INBCNT"}\NormalTok{,}
    \StringTok{"X12MOMAR"}\NormalTok{,}
    \StringTok{"X1NUMSIB"}\NormalTok{, }
    \StringTok{"P1OLDMOM"}\NormalTok{,}
    \StringTok{"P1CHLDBK"}\NormalTok{,}
    \StringTok{"P2DISTHM"}\NormalTok{,}
    \StringTok{"P1NUMPLA"}\NormalTok{,}
    \StringTok{"T2PARIN"}\NormalTok{,}
    \StringTok{"X12PAR1ED\_I"}\NormalTok{,}
    \StringTok{"X12PAR2ED\_I"}\NormalTok{,}
    \StringTok{"X2INCCAT\_I"}\NormalTok{,}
    \StringTok{"X1PAR1EMP"}\NormalTok{,}
    \StringTok{"S2LUNCH"}\NormalTok{,}
    \StringTok{"X2KRCETH"}\NormalTok{,}
    \StringTok{"S2NGHBOR"}\NormalTok{,}
    \StringTok{"S2OUTSID"}\NormalTok{,}
    \StringTok{"S2USDABR"}\NormalTok{,}
    \StringTok{"S2PUBSOC"}\NormalTok{,}
    \StringTok{"X1LOCALE"}\NormalTok{,}
    \StringTok{"S1\_ID"}\NormalTok{,}
    \StringTok{"W1\_2P0PSU"}\NormalTok{,}
    \StringTok{"prop.missing"}\NormalTok{)}

\NormalTok{covariateNamesBART }\OtherTok{\textless{}{-}} \FunctionTok{c}\NormalTok{(}
    \StringTok{"X1RTHETK1"}\NormalTok{, }
    \StringTok{"X1MTHETK1"}\NormalTok{,}
    \StringTok{"X1TCHAPP"}\NormalTok{, }
    \StringTok{"X1TCHCON"}\NormalTok{,}
    \StringTok{"X1TCHPER"}\NormalTok{, }
    \StringTok{"X1TCHEXT"}\NormalTok{,}
    \StringTok{"X1TCHINT"}\NormalTok{,}
    \StringTok{"X1ATTNFS"}\NormalTok{,}
    \StringTok{"X1INBCNT"}\NormalTok{,}
    \StringTok{"X12MOMAR"}\NormalTok{,}
    \StringTok{"X1NUMSIB"}\NormalTok{, }
    \StringTok{"P1OLDMOM"}\NormalTok{,}
    \StringTok{"P1CHLDBK"}\NormalTok{,}
    \StringTok{"P2DISTHM"}\NormalTok{,}
    \StringTok{"P1NUMPLA"}\NormalTok{,}
    \StringTok{"T2PARIN"}\NormalTok{,}
    \StringTok{"X12PAR1ED\_I"}\NormalTok{,}
    \StringTok{"X12PAR2ED\_I"}\NormalTok{,}
    \StringTok{"X2INCCAT\_I"}\NormalTok{,}
    \StringTok{"X1PAR1EMP"}\NormalTok{,}
    \StringTok{"S2LUNCH"}\NormalTok{,}
    \StringTok{"X2KRCETH"}\NormalTok{,}
    \StringTok{"S2NGHBOR"}\NormalTok{,}
    \StringTok{"S2OUTSID"}\NormalTok{,}
    \StringTok{"S2USDABR"}\NormalTok{,}
    \StringTok{"S2PUBSOC"}\NormalTok{,}
    \StringTok{"X1LOCALE"}\NormalTok{)}

\NormalTok{psFormula }\OtherTok{\textless{}{-}} \FunctionTok{paste}\NormalTok{(covariateNames, }\AttributeTok{collapse=}\StringTok{"+"}\NormalTok{)}
\NormalTok{psFormula }\OtherTok{\textless{}{-}} \FunctionTok{formula}\NormalTok{(}\FunctionTok{paste}\NormalTok{(}\StringTok{"treated\textasciitilde{}"}\NormalTok{, psFormula, }\AttributeTok{sep=}\StringTok{""}\NormalTok{))}
\FunctionTok{print}\NormalTok{(psFormula) }
\end{Highlighting}
\end{Shaded}

\begin{verbatim}
## treated ~ X1RTHETK1 + X1MTHETK1 + X1TCHAPP + X1TCHCON + X1TCHPER + 
##     X1TCHEXT + X1TCHINT + X1ATTNFS + X1INBCNT + X12MOMAR + X1NUMSIB + 
##     P1OLDMOM + P1CHLDBK + P2DISTHM + P1NUMPLA + T2PARIN + X12PAR1ED_I + 
##     X12PAR2ED_I + X2INCCAT_I + X1PAR1EMP + S2LUNCH + X2KRCETH + 
##     S2NGHBOR + S2OUTSID + S2USDABR + S2PUBSOC + X1LOCALE + S1_ID + 
##     W1_2P0PSU + prop.missing
\end{verbatim}

\hypertarget{estimating-cates-using-machine-learning-methods}{%
\section{Estimating CATEs Using Machine Learning
Methods}\label{estimating-cates-using-machine-learning-methods}}

\hypertarget{bart}{%
\subsection{BART}\label{bart}}

\begin{Shaded}
\begin{Highlighting}[]
\CommentTok{\# estimating conditional average treatment effects (CATEs) using BART}

\NormalTok{bart }\OtherTok{\textless{}{-}} \FunctionTok{bartc}\NormalTok{(}\AttributeTok{response =}\NormalTok{ train}\SpecialCharTok{$}\NormalTok{T2TTABS,}
             \AttributeTok{treatment =}\NormalTok{ train}\SpecialCharTok{$}\NormalTok{treated,}
             \AttributeTok{confounders =} \FunctionTok{data.frame}\NormalTok{(train[,covariateNamesBART]),}
             \AttributeTok{method.rsp =} \StringTok{"bart"}\NormalTok{, }
             \AttributeTok{method.trt =} \StringTok{"bart"}\NormalTok{,}
             \AttributeTok{keepTrees =} \ConstantTok{TRUE}\NormalTok{,}
             \AttributeTok{estimand =} \StringTok{"ate"}\NormalTok{)}
\end{Highlighting}
\end{Shaded}

\begin{verbatim}
## fitting treatment model via method 'bart'
## fitting response model via method 'bart'
\end{verbatim}

\begin{Shaded}
\begin{Highlighting}[]
\NormalTok{cate }\OtherTok{\textless{}{-}} \FunctionTok{predict}\NormalTok{(bart,}
               \AttributeTok{newdata =} \FunctionTok{data.frame}\NormalTok{(test[,covariateNamesBART]),}
               \AttributeTok{type =} \StringTok{"icate"}\NormalTok{)}

\NormalTok{cate\_m }\OtherTok{\textless{}{-}} \FunctionTok{apply}\NormalTok{(cate, }\DecValTok{2}\NormalTok{, mean) }
\end{Highlighting}
\end{Shaded}

\hypertarget{genericml}{%
\subsection{GenericML}\label{genericml}}

\begin{Shaded}
\begin{Highlighting}[]
\DocumentationTok{\#\#\textquotesingle{} @Matt}

\NormalTok{learners }\OtherTok{\textless{}{-}} \FunctionTok{c}\NormalTok{(}\StringTok{"random\_forest"}\NormalTok{, }\StringTok{"lasso"}\NormalTok{)}

\NormalTok{matrix\_covs }\OtherTok{\textless{}{-}} \FunctionTok{as.matrix}\NormalTok{(data }\SpecialCharTok{\%\textgreater{}\%} \FunctionTok{select}\NormalTok{(}\FunctionTok{all\_of}\NormalTok{(covariateNames)) }\SpecialCharTok{\%\textgreater{}\%}
                           \FunctionTok{mutate}\NormalTok{(}\FunctionTok{across}\NormalTok{(}\AttributeTok{.fns =}\NormalTok{ as.numeric))) }

\NormalTok{X1 }\OtherTok{\textless{}{-}} \FunctionTok{setup\_X1}\NormalTok{(}\AttributeTok{funs\_Z =} \FunctionTok{c}\NormalTok{(}\StringTok{"B"}\NormalTok{, }\StringTok{"S"}\NormalTok{))}
               \CommentTok{\#fixed\_effects = vil\_pair)}

\NormalTok{vcov }\OtherTok{\textless{}{-}} \FunctionTok{setup\_vcov}\NormalTok{(}\AttributeTok{estimator =} \StringTok{"vcovHC"}\NormalTok{)}
                   \CommentTok{\#arguments = list(cluster = demi\_paire))}

\FunctionTok{library}\NormalTok{(parsnip)}
\NormalTok{ps\_nnet }\OtherTok{\textless{}{-}} \FunctionTok{mlp}\NormalTok{(}\AttributeTok{mode =} \StringTok{"classification"}\NormalTok{,}
               \AttributeTok{engine =} \StringTok{"nnet"}\NormalTok{,}
\AttributeTok{hidden\_units =} \DecValTok{20}\NormalTok{) }\SpecialCharTok{\%\textgreater{}\%}
  \FunctionTok{fit}\NormalTok{(psFormula,}
      \AttributeTok{data =}\NormalTok{ data)}
\NormalTok{data}\SpecialCharTok{$}\NormalTok{ps\_nnet }\OtherTok{\textless{}{-}} \FunctionTok{predict}\NormalTok{(ps\_nnet,}
                      \AttributeTok{new\_data =}\NormalTok{ data,}
                      \AttributeTok{type =} \StringTok{"prob"}\NormalTok{)[,}\DecValTok{2}\NormalTok{]}
\NormalTok{data}\SpecialCharTok{$}\NormalTok{ps\_nnet }\OtherTok{\textless{}{-}}\NormalTok{ data}\SpecialCharTok{$}\NormalTok{ps\_nnet}\SpecialCharTok{$}\NormalTok{.pred\_1 }\DocumentationTok{\#\# remove the $column.name}

\NormalTok{genML }\OtherTok{\textless{}{-}} \FunctionTok{GenericML}\NormalTok{(}
  \AttributeTok{Z =}\NormalTok{ matrix\_covs, }\CommentTok{\#covariates}
  \AttributeTok{D =} \FunctionTok{as.numeric}\NormalTok{(}\FunctionTok{as.character}\NormalTok{(data}\SpecialCharTok{$}\NormalTok{treated)), }\CommentTok{\#treatment}
  \AttributeTok{Y =} \FunctionTok{as.numeric}\NormalTok{(data}\SpecialCharTok{$}\NormalTok{T2TTABS), }\CommentTok{\#outcome}
  \AttributeTok{learners\_GenericML =}\NormalTok{ learners,  }\CommentTok{\# learners specified above}
  \AttributeTok{learner\_propensity\_score =} \FunctionTok{as.numeric}\NormalTok{(data}\SpecialCharTok{$}\NormalTok{ps\_nnet), }\CommentTok{\#as.numeric(data$ps)  \#ps}
  \AttributeTok{num\_splits =} \DecValTok{10}\NormalTok{,                        }\CommentTok{\# number splits of the data}
  \AttributeTok{quantile\_cutoffs =} \FunctionTok{c}\NormalTok{(}\FloatTok{0.2}\NormalTok{, }\FloatTok{0.4}\NormalTok{, }\FloatTok{0.6}\NormalTok{, }\FloatTok{0.8}\NormalTok{), }\CommentTok{\# grouping for CATEs}
  \AttributeTok{significance\_level =} \FloatTok{0.05}\NormalTok{,                }\CommentTok{\# significance level}
  \AttributeTok{X1\_BLP =}\NormalTok{ X1, }\AttributeTok{X1\_GATES =}\NormalTok{ X1,               }\CommentTok{\# regression setup}
  \AttributeTok{vcov\_BLP =}\NormalTok{ vcov, }\AttributeTok{vcov\_GATES =}\NormalTok{ vcov,       }\CommentTok{\# covariance setup}
  \AttributeTok{parallel =}\NormalTok{ F, }\CommentTok{\#num\_cores = 6L, \# parallelization}
  \AttributeTok{seed =} \DecValTok{20220621}\NormalTok{)                         }\CommentTok{\# RNG seed}

\NormalTok{best }\OtherTok{\textless{}{-}} \FunctionTok{get\_best}\NormalTok{(genML)}
\DocumentationTok{\#\# random\_forest is best, becomes the default for all future GenML functions}

\NormalTok{base }\OtherTok{\textless{}{-}} \FunctionTok{get\_BLP}\NormalTok{(genML)}
\DocumentationTok{\#\# Β2 is significant indicating treatment heterogeneity}

\NormalTok{a }\OtherTok{\textless{}{-}} \FunctionTok{get\_GATES}\NormalTok{(genML) }\SpecialCharTok{\%\textgreater{}\%}
  \FunctionTok{plot}\NormalTok{()}


\CommentTok{\# Plot parental education}
\NormalTok{b }\OtherTok{\textless{}{-}} \FunctionTok{get\_CLAN}\NormalTok{(genML,}
         \AttributeTok{variable =} \StringTok{"X12PAR1ED\_I"}\NormalTok{) }\SpecialCharTok{\%\textgreater{}\%}
  \FunctionTok{plot}\NormalTok{() }\SpecialCharTok{+}
  \FunctionTok{labs}\NormalTok{(}\AttributeTok{title =} \StringTok{"Heterogeneity by First Parents Education"}\NormalTok{,}
       \AttributeTok{y =} \FunctionTok{str\_wrap}\NormalTok{(}\StringTok{"Average Value of Parental Education"}\NormalTok{, }\DecValTok{25}\NormalTok{))}

\CommentTok{\# Plot family marriage status}
\NormalTok{c }\OtherTok{\textless{}{-}} \FunctionTok{get\_CLAN}\NormalTok{(genML,}
         \AttributeTok{variable =} \StringTok{"X12MOMAR"}\NormalTok{) }\SpecialCharTok{\%\textgreater{}\%}
  \FunctionTok{plot}\NormalTok{() }\SpecialCharTok{+}
  \FunctionTok{labs}\NormalTok{(}\AttributeTok{title =} \StringTok{"Heterogeneity by Parental Marriage Status"}\NormalTok{,}
       \AttributeTok{y =} \FunctionTok{str\_wrap}\NormalTok{(}\StringTok{"Average Value of Parental Marriage Status"}\NormalTok{, }\DecValTok{25}\NormalTok{))}


\CommentTok{\# Plot lunch variable}
\NormalTok{d }\OtherTok{\textless{}{-}} \FunctionTok{get\_CLAN}\NormalTok{(genML,}
         \AttributeTok{variable =} \StringTok{"S2LUNCH"}\NormalTok{) }\SpecialCharTok{\%\textgreater{}\%}
  \FunctionTok{plot}\NormalTok{() }\SpecialCharTok{+}
  \FunctionTok{labs}\NormalTok{(}\AttributeTok{title =} \StringTok{"Heterogeneity by Free School Lunch"}\NormalTok{,}
       \AttributeTok{y =} \FunctionTok{str\_wrap}\NormalTok{(}\StringTok{"Average Value of Students Receiving Free Lunch"}\NormalTok{, }\DecValTok{25}\NormalTok{))}

\CommentTok{\# Plot percent coming from neighborhood}
\NormalTok{e }\OtherTok{\textless{}{-}} \FunctionTok{get\_CLAN}\NormalTok{(genML,}
         \AttributeTok{variable =} \StringTok{"S2NGHBOR"}\NormalTok{) }\SpecialCharTok{\%\textgreater{}\%}
  \FunctionTok{plot}\NormalTok{() }\SpecialCharTok{+}
  \FunctionTok{labs}\NormalTok{(}\AttributeTok{title =} \StringTok{"Heterogeneity by Percent of School coming from }
\StringTok{       Surrounding Neighborhood"}\NormalTok{,}
       \AttributeTok{y =} \FunctionTok{str\_wrap}\NormalTok{(}\StringTok{"Average Value of Percent Coming from Neighborhood"}\NormalTok{, }\DecValTok{25}\NormalTok{))}


\FunctionTok{library}\NormalTok{(patchwork)}
\NormalTok{layout }\OtherTok{\textless{}{-}} \FunctionTok{c}\NormalTok{(}\StringTok{"\#AA\#}
\StringTok{            BBCC}
\StringTok{            DDEE"}\NormalTok{)}
\NormalTok{genMLplot}\OtherTok{\textless{}{-}}\NormalTok{ a }\SpecialCharTok{+}\NormalTok{ b }\SpecialCharTok{+}\NormalTok{ c }\SpecialCharTok{+}\NormalTok{ d }\SpecialCharTok{+}\NormalTok{ e }\SpecialCharTok{+}
  \FunctionTok{plot\_layout}\NormalTok{(}\AttributeTok{design =}\NormalTok{ layout)}
\end{Highlighting}
\end{Shaded}

\newpage

\begin{Shaded}
\begin{Highlighting}[]
\NormalTok{genMLplot}
\end{Highlighting}
\end{Shaded}

\includegraphics{data_analysis_2_files/figure-latex/unnamed-chunk-1-1.pdf}
\newpage \#\# causal forests

\begin{Shaded}
\begin{Highlighting}[]
\CommentTok{\#the grf package only takes numeric covariates}
\NormalTok{data2 }\OtherTok{\textless{}{-}}\NormalTok{ data}
\CommentTok{\#So convert those factor variables to be the numeric class}
\ControlFlowTok{for}\NormalTok{ (i }\ControlFlowTok{in} \DecValTok{1}\SpecialCharTok{:}\FunctionTok{length}\NormalTok{(covariateNames)) \{}
  \ControlFlowTok{if}\NormalTok{(}\FunctionTok{class}\NormalTok{(data2[,covariateNames[i]])}\SpecialCharTok{==}\StringTok{"factor"}\NormalTok{)\{}
\NormalTok{    data2[, covariateNames[i]] }\OtherTok{\textless{}{-}} \FunctionTok{as.numeric}\NormalTok{(}\FunctionTok{as.character}\NormalTok{(data2[,covariateNames[i]]))}
\NormalTok{  \}}
\NormalTok{\}}

\CommentTok{\#Step 1: Split data into training data set and testing data set}
\CommentTok{\#In this case, we split it to be 50/50}
\FunctionTok{set.seed}\NormalTok{(}\DecValTok{123}\NormalTok{)}
\NormalTok{train\_index }\OtherTok{\textless{}{-}} \FunctionTok{sample}\NormalTok{(}\DecValTok{1}\SpecialCharTok{:}\FunctionTok{nrow}\NormalTok{(data2), }\FunctionTok{nrow}\NormalTok{(data2)}\SpecialCharTok{/}\DecValTok{2}\NormalTok{)}
\NormalTok{train\_index }\OtherTok{\textless{}{-}}\NormalTok{ train\_index[}\FunctionTok{order}\NormalTok{(train\_index)]}

\NormalTok{train\_data }\OtherTok{\textless{}{-}}\NormalTok{ data2[train\_index,]}
\NormalTok{test\_data }\OtherTok{\textless{}{-}}\NormalTok{ data2[}\SpecialCharTok{{-}}\NormalTok{train\_index,]}

\CommentTok{\#Step 2: model fit, using causal forest}
\CommentTok{\#Tuning mtry and min.node.size parameters by setting tune.parameters}
\NormalTok{train.forest }\OtherTok{=} \FunctionTok{causal\_forest}\NormalTok{(}\AttributeTok{X=}\NormalTok{train\_data[,covariateNames],}
                              \AttributeTok{Y =}\NormalTok{ train\_data}\SpecialCharTok{$}\NormalTok{X2MTHETK1, }\AttributeTok{num.trees =} \DecValTok{5000}\NormalTok{,}
                              \AttributeTok{W =} \FunctionTok{as.numeric}\NormalTok{(}\FunctionTok{as.character}\NormalTok{(train\_data}\SpecialCharTok{$}\NormalTok{treated)),}
                              \AttributeTok{W.hat =}\NormalTok{ train\_data}\SpecialCharTok{$}\NormalTok{ps,}
                              \AttributeTok{tune.parameters =} \FunctionTok{c}\NormalTok{(}\StringTok{"mtry"}\NormalTok{, }\StringTok{"min.node.size"}\NormalTok{),}
                              \AttributeTok{seed =} \DecValTok{0}\NormalTok{)}

\NormalTok{train.forest[[}\StringTok{"tuning.output"}\NormalTok{]]}
\end{Highlighting}
\end{Shaded}

\begin{verbatim}
## Tuning status: tuned.
## This indicates tuning found parameters that are expected to perform better than default. 
## 
## Predicted debiased error: 0.192235789762278
## 
## Tuned parameters: 
## mtry: 12
##  min.node.size: 1
## 
## Average error by 5-quantile:
## 
##     mtry     error
##    [1,6] 0.1932963
##   (6,11] 0.1932316
##  (11,16] 0.1932540
##  (16,21] 0.1932606
##  (21,26] 0.1931298
## 
##  min.node.size     error
##          [1,2] 0.1920488
##          (2,9] 0.1926641
##       (9,32.4] 0.1932746
##     (32.4,124] 0.1939208
##      (124,393] 0.1943589
\end{verbatim}

\begin{Shaded}
\begin{Highlighting}[]
\CommentTok{\#The results showed that mtry = 16 and min.node.size = 1 perform better than the default setting}

\CommentTok{\#Step 3: Obtain estimates of the conditional average treatment effect (CATE)}
\CommentTok{\#with standard errors}
\NormalTok{tau.hat }\OtherTok{=} \FunctionTok{predict}\NormalTok{(train.forest,}\AttributeTok{X=}\NormalTok{ test\_data[,covariateNames], }\AttributeTok{estimate.variance =}\NormalTok{ T)}
\NormalTok{CATE\_causalForest }\OtherTok{=}\NormalTok{ tau.hat}\SpecialCharTok{$}\NormalTok{predictions}
\end{Highlighting}
\end{Shaded}

\hypertarget{figures-for-three-separate-methods}{%
\section{Figures for Three Separate
Methods}\label{figures-for-three-separate-methods}}

\begin{Shaded}
\begin{Highlighting}[]
\CommentTok{\# \# BART}
\CommentTok{\# ggplot(data, aes(x=logit\_scores, color = selfEmploy)) + }
\CommentTok{\#   geom\_boxplot() + ggtitle("Logit Regression")}
\CommentTok{\# }
\CommentTok{\# \# GenericML}
\CommentTok{\# ggplot(data, aes(x=forest\_scores, color = selfEmploy)) + }
\CommentTok{\#   geom\_boxplot() + ggtitle("Random Forest")}
\CommentTok{\# }
\CommentTok{\# ggplot(data, aes(x=GBM\_scores, color = selfEmploy)) + }
\CommentTok{\#   geom\_boxplot() + ggtitle("GBM")}

\CommentTok{\# Causal Forests}
\CommentTok{\#1. correlation matrix}
\CommentTok{\#causal forest only output the best tunning parameters\textquotesingle{} model fit outcomes}
\CommentTok{\#To answer Q2, I run one more model fit with mtry = 4 and min.node.size = 50}
\NormalTok{train.forest2 }\OtherTok{=} \FunctionTok{causal\_forest}\NormalTok{(}\AttributeTok{X=}\NormalTok{train\_data[,covariateNames],}
                              \AttributeTok{Y =}\NormalTok{ train\_data}\SpecialCharTok{$}\NormalTok{X2MTHETK1, }\AttributeTok{num.trees =} \DecValTok{5000}\NormalTok{,}
                              \AttributeTok{W =} \FunctionTok{as.numeric}\NormalTok{(}\FunctionTok{as.character}\NormalTok{(train\_data}\SpecialCharTok{$}\NormalTok{treated)),}
                              \AttributeTok{W.hat =}\NormalTok{ train\_data}\SpecialCharTok{$}\NormalTok{ps,}
                              \AttributeTok{mtry =} \DecValTok{4}\NormalTok{, }\AttributeTok{min.node.size =} \DecValTok{50}\NormalTok{,}
                              \AttributeTok{seed =} \DecValTok{0}\NormalTok{)}
\NormalTok{tau.hat2 }\OtherTok{=} \FunctionTok{predict}\NormalTok{(train.forest2,}\AttributeTok{X=}\NormalTok{ test\_data[,covariateNames], }\AttributeTok{estimate.variance =}\NormalTok{ T)}
\NormalTok{CATE2\_causalForest }\OtherTok{=}\NormalTok{ tau.hat2}\SpecialCharTok{$}\NormalTok{predictions}

\FunctionTok{cor}\NormalTok{(CATE\_causalForest, CATE2\_causalForest)}
\end{Highlighting}
\end{Shaded}

\begin{verbatim}
## [1] 0.7788822
\end{verbatim}

\begin{Shaded}
\begin{Highlighting}[]
\CommentTok{\#2.box plot}
\FunctionTok{boxplot}\NormalTok{(CATE\_causalForest, CATE2\_causalForest)}
\end{Highlighting}
\end{Shaded}

\includegraphics{data_analysis_2_files/figure-latex/q_2-1.pdf}

\begin{Shaded}
\begin{Highlighting}[]
\CommentTok{\#3.QQ plot}
\FunctionTok{qqplot}\NormalTok{(CATE\_causalForest, CATE2\_causalForest)}
\end{Highlighting}
\end{Shaded}

\includegraphics{data_analysis_2_files/figure-latex/q_2-2.pdf}

\hypertarget{determine-best-linear-projection-of-catevariable-importance}{%
\section{Determine Best Linear Projection of CATE/Variable
Importance}\label{determine-best-linear-projection-of-catevariable-importance}}

\begin{Shaded}
\begin{Highlighting}[]
\CommentTok{\#BART}


\CommentTok{\#GenericML}


\CommentTok{\#Causal Forests}
\CommentTok{\#Step 1: Subset important variables}
\NormalTok{importance\_cf }\OtherTok{=} \FunctionTok{variable\_importance}\NormalTok{(train.forest)}
\FunctionTok{rownames}\NormalTok{(importance\_cf) }\OtherTok{=} \FunctionTok{names}\NormalTok{(train\_data[,covariateNames])}

\CommentTok{\#select variables above the median of importance of the aggregated importances}
\CommentTok{\#across imputed datasets}
\NormalTok{important.var\_cf }\OtherTok{=} \FunctionTok{rownames}\NormalTok{(importance\_cf)[importance\_cf}\SpecialCharTok{\textgreater{}}\FunctionTok{median}\NormalTok{(importance\_cf)]}

\CommentTok{\#Step 2}
\CommentTok{\#run test forest}
\NormalTok{test.forest }\OtherTok{=} \FunctionTok{causal\_forest}\NormalTok{(}\AttributeTok{X =}\NormalTok{ test\_data[,important.var\_cf],}
                            \AttributeTok{Y =}\NormalTok{ test\_data}\SpecialCharTok{$}\NormalTok{X2MTHETK1,}
                            \AttributeTok{W =} \FunctionTok{as.numeric}\NormalTok{(}\FunctionTok{as.character}\NormalTok{(test\_data}\SpecialCharTok{$}\NormalTok{treated)),}
                            \AttributeTok{W.hat =}\NormalTok{ test\_data}\SpecialCharTok{$}\NormalTok{ps,}
                            \AttributeTok{mtry =} \DecValTok{16}\NormalTok{, }\AttributeTok{num.trees=}\DecValTok{5000}\NormalTok{,}
                            \AttributeTok{min.node.size =} \DecValTok{1}\NormalTok{, }\AttributeTok{seed =} \DecValTok{0}\NormalTok{)}

\CommentTok{\#Step 3: Predict the conditional average treatment effect (CATE)}
\NormalTok{tau.hat }\OtherTok{=} \FunctionTok{predict}\NormalTok{(test.forest,}\AttributeTok{X=}\NormalTok{ test\_data[,important.var\_cf], }\AttributeTok{estimate.variance =}\NormalTok{ T)}
\NormalTok{CATE\_test }\OtherTok{=}\NormalTok{ tau.hat}\SpecialCharTok{$}\NormalTok{predictions}
\FunctionTok{summary}\NormalTok{(CATE\_test)}
\end{Highlighting}
\end{Shaded}

\begin{verbatim}
##      Min.   1st Qu.    Median      Mean   3rd Qu.      Max. 
## -0.190572 -0.016740  0.009800  0.008658  0.035878  0.265934
\end{verbatim}

\begin{Shaded}
\begin{Highlighting}[]
\CommentTok{\#When using the test data set and fitting the important variables,}
 \CommentTok{\#the conditional ATE was small, which ranges from {-}0.401 to 0.254 with a mean of 0.018.}

\FunctionTok{test\_calibration}\NormalTok{(test.forest)}
\end{Highlighting}
\end{Shaded}

\begin{verbatim}
## 
## Best linear fit using forest predictions (on held-out data)
## as well as the mean forest prediction as regressors, along
## with one-sided heteroskedasticity-robust (HC3) SEs:
## 
##                                Estimate Std. Error t value Pr(>t)
## mean.forest.prediction          1.33876    1.52512  0.8778 0.1900
## differential.forest.prediction  0.24307    0.42471  0.5723 0.2836
\end{verbatim}

\begin{Shaded}
\begin{Highlighting}[]
\CommentTok{\#The outcomes showed that the coefficient of the mean forest prediction was 1 }
 \CommentTok{\#which indicated the mean forest prediction was correct. }
\CommentTok{\#Also, the results indicated no heterogeneity been detected.}
\end{Highlighting}
\end{Shaded}

\hypertarget{two-most-important-cate-predictors-for-each-method-plot-catepredictor-relationship}{%
\section{Two most important CATE predictors (For each method), plot
CATE/predictor
relationship}\label{two-most-important-cate-predictors-for-each-method-plot-catepredictor-relationship}}

\begin{Shaded}
\begin{Highlighting}[]
\CommentTok{\#BART}


\CommentTok{\#GenericML}


\CommentTok{\#Causal Forests}
\CommentTok{\#The most two predictors are}
\NormalTok{Top2predictors }\OtherTok{\textless{}{-}}\NormalTok{ importance\_cf[}\FunctionTok{order}\NormalTok{(importance\_cf, }\AttributeTok{decreasing =}\NormalTok{ T),][}\DecValTok{1}\SpecialCharTok{:}\DecValTok{2}\NormalTok{]}

\CommentTok{\#Predictor 1}
\NormalTok{groupX1MTHETK1 }\OtherTok{\textless{}{-}} \FunctionTok{quantile}\NormalTok{(test\_data[, }\FunctionTok{names}\NormalTok{(Top2predictors)[}\DecValTok{1}\NormalTok{]])}
\NormalTok{test\_data}\SpecialCharTok{$}\NormalTok{groupX1M }\OtherTok{\textless{}{-}} \FunctionTok{ifelse}\NormalTok{(test\_data}\SpecialCharTok{$}\NormalTok{X1MTHETK1 }\SpecialCharTok{\textgreater{}=}\NormalTok{ groupX1MTHETK1[}\DecValTok{4}\NormalTok{], }\DecValTok{4}\NormalTok{, }
                             \FunctionTok{ifelse}\NormalTok{(test\_data}\SpecialCharTok{$}\NormalTok{X1MTHETK1}\SpecialCharTok{\textgreater{}=}\NormalTok{groupX1MTHETK1[}\DecValTok{3}\NormalTok{] }\SpecialCharTok{\&}\NormalTok{ test\_data}\SpecialCharTok{$}\NormalTok{X1MTHETK1}\SpecialCharTok{\textless{}}\NormalTok{groupX1MTHETK1[}\DecValTok{4}\NormalTok{], }\DecValTok{3}\NormalTok{,}
                                    \FunctionTok{ifelse}\NormalTok{(test\_data}\SpecialCharTok{$}\NormalTok{X1MTHETK1}\SpecialCharTok{\textgreater{}=}\NormalTok{groupX1MTHETK1[}\DecValTok{2}\NormalTok{] }\SpecialCharTok{\&}\NormalTok{ test\_data}\SpecialCharTok{$}\NormalTok{X1MTHETK1}\SpecialCharTok{\textless{}}\NormalTok{groupX1MTHETK1[}\DecValTok{3}\NormalTok{], }\DecValTok{2}\NormalTok{,}
                                           \DecValTok{1}\NormalTok{)))}
\NormalTok{test\_data}\SpecialCharTok{$}\NormalTok{groupX1M }\OtherTok{\textless{}{-}} \FunctionTok{factor}\NormalTok{(test\_data}\SpecialCharTok{$}\NormalTok{groupX1M)}
\FunctionTok{boxplot}\NormalTok{(CATE\_causalForest }\SpecialCharTok{\textasciitilde{}}\NormalTok{ test\_data}\SpecialCharTok{$}\NormalTok{groupX1M, }\AttributeTok{xlab =} \StringTok{"X1MTHETK1 Group"}\NormalTok{, }\AttributeTok{ylab =} \StringTok{"CATE"}\NormalTok{)}
\end{Highlighting}
\end{Shaded}

\includegraphics{data_analysis_2_files/figure-latex/q_4-1.pdf}

\begin{Shaded}
\begin{Highlighting}[]
\CommentTok{\#Predictor 2}
\NormalTok{groupX1R }\OtherTok{\textless{}{-}} \FunctionTok{quantile}\NormalTok{(test\_data[, }\FunctionTok{names}\NormalTok{(Top2predictors)[}\DecValTok{2}\NormalTok{]])}
\NormalTok{test\_data}\SpecialCharTok{$}\NormalTok{groupX1R }\OtherTok{\textless{}{-}} \FunctionTok{ifelse}\NormalTok{(test\_data}\SpecialCharTok{$}\NormalTok{X1RTHETK1 }\SpecialCharTok{\textgreater{}=}\NormalTok{ groupX1R[}\DecValTok{4}\NormalTok{], }\DecValTok{4}\NormalTok{, }
                             \FunctionTok{ifelse}\NormalTok{(test\_data}\SpecialCharTok{$}\NormalTok{X1RTHETK1}\SpecialCharTok{\textgreater{}=}\NormalTok{groupX1R[}\DecValTok{3}\NormalTok{] }\SpecialCharTok{\&}\NormalTok{ test\_data}\SpecialCharTok{$}\NormalTok{X1RTHETK1}\SpecialCharTok{\textless{}}\NormalTok{groupX1R[}\DecValTok{4}\NormalTok{], }\DecValTok{3}\NormalTok{,}
                                    \FunctionTok{ifelse}\NormalTok{(test\_data}\SpecialCharTok{$}\NormalTok{X1RTHETK1}\SpecialCharTok{\textgreater{}=}\NormalTok{groupX1R[}\DecValTok{2}\NormalTok{] }\SpecialCharTok{\&}\NormalTok{ test\_data}\SpecialCharTok{$}\NormalTok{X1RTHETK1}\SpecialCharTok{\textless{}}\NormalTok{groupX1R[}\DecValTok{3}\NormalTok{], }\DecValTok{2}\NormalTok{,}
                                           \DecValTok{1}\NormalTok{)))}
\NormalTok{test\_data}\SpecialCharTok{$}\NormalTok{groupX1R }\OtherTok{\textless{}{-}} \FunctionTok{factor}\NormalTok{(test\_data}\SpecialCharTok{$}\NormalTok{groupX1R)}
\FunctionTok{boxplot}\NormalTok{(CATE\_causalForest }\SpecialCharTok{\textasciitilde{}}\NormalTok{ test\_data}\SpecialCharTok{$}\NormalTok{groupX1R, }\AttributeTok{xlab =} \StringTok{"X1RTHETK1 Group"}\NormalTok{, }\AttributeTok{ylab =} \StringTok{"CATE"}\NormalTok{)}
\end{Highlighting}
\end{Shaded}

\includegraphics{data_analysis_2_files/figure-latex/q_4-2.pdf}

\end{document}
